% !TEX program = xelatex
% this file is encoded in utf-8
% v1.7

\documentclass[12pt, a4paper]{ntust_report} 

\usepackage{amssymb}
%\usepackage{libertine}
%\usepackage[T1]{fontenc}
%\usepackage{lmodern}
%\usepackage{kpfonts}
%\usepackage[sc]{mathpazo}
%\usepackage{microtype}
%\usepackage{slantsc}% http://ctan.org/pkg/slantsc

\usepackage{fontspec}	% To enable font settings 加這個就可以設定字體 
\usepackage{xeCJK}      % To let Chinese-English font set in separate location 讓中英文字體分開設置

%\usepackage[indentfirst=true]{xeCJK}
%\setmainfont{Arial}            %設定主要字型,也就是英文字型 
\setmainfont[Mapping=tex-text]{Times New Roman}
%\setmainfont{Linux Libertine O}
%\setmainfont{Junicode}
%\setCJKmainfont{DFKai-SB}     %設定中文字型 
%\setCJKmainfont{BiauKai}

\setCJKmainfont{AR PL KaitiM Big5}
\XeTeXlinebreaklocale "zh"      		% These two lines have to be added, 這兩行一定要加,中文才能自動換行 
\XeTeXlinebreakskip = 0pt plus 1pt      % 	so that the Chinese sentence can change line automatically.



% Unless school modifies the thesis format (margins, header, footer, watermark)
% or needs to improve the use of LaTeX packages,
% or needs to change the default fonts or the encoding,
% then there is no need to modify the content inside this file.
% For thesis writing personalization, please modify the files starting with 'my_'

\usepackage[nospace]{cite}  		% for smart citation
\usepackage{geometry}  				% for easy margin settings
%\usepackage{subfigure}  			% for subfigure
%\usepackage[dvipdfm]{graphicx}  	% for graphic using eps
%\usepackage[xetex]{graphicx}
%\usepackage{graphicx}  			% for graphic using eps

\usepackage{epstopdf} 				% for automatic transformation (eps to pdf) while using pdflatex

%\usepackage{algorithmic}  			% for algorithm
\usepackage{algorithm}
\usepackage{algorithmicx}
\usepackage{algpseudocode}

%
% margins setting
\geometry{verbose,a4paper,tmargin=3.5cm,bmargin=2cm,lmargin=3cm,rmargin=3cm}
%
\usepackage{amsmath} 				% for mathematical function (AMS Format)
%\usepackage{amssymb} 				% for mathematical symbol (AMS Format)
										% if 'amssymb' and 'Xunicode' conflict, please put 'amssymb' beforehand
\usepackage{mathrsfs} 				% for cursive mathematical symbol,
										% for example, typing '\mathscr{E} may result 'cursive E'
\usepackage{listings} 				% for listing any programming code
%
% listing setting
\lstset{breaklines=true,			% a long line of code can be broken into lines of code
extendedchars=false,				
texcl=true,							
comment=[l]\%\%,					
basicstyle=\small,					% approximately 10 pt
commentstyle=\upshape,				% the default is 'italics', so set it upright
%language=Octave 					% some 'octave' instruction will appear in bold
}

\usepackage{url} 					% for trackbacking URL in the document; usage: \url{http://www.ntust.edu.tw}

% Illustration package: graphicx
	% whether users using 'pdftex' or 'latex + dvipdfmx' for their workflow?
	% each case may have different parameters
	% the following automatic decision
	% refered from
	% http://www.tex.ac.uk/cgi-bin/texfaq2html?label=ifpdf

%\newcommand\mydvipdfmxflow{dvipdfmx}
%\newcommand\mypdftexflow{pdftex}
%
%\ifx\pdfoutput\undefined
%  % not running pdftex
%  \usepackage[dvipdfm]{graphicx}
%  \newcommand\myworkflow{dvipdfmx}  % set the flag for hyperref
%\else
%  \ifx\pdfoutput\relax
%    % not running pdftex
%    \usepackage[dvipdfm]{graphicx}
%    \newcommand\myworkflow{dvipdfmx}  % set the flag
%  \else
%    % running pdftex, with...
%    \ifnum\pdfoutput>0
%      % ... PDF output
%      \usepackage[pdftex]{graphicx}
%      \newcommand\myworkflow{pdftex}  % set the flag
%    \else
%      %...DVI output
%      \usepackage[dvipdfm]{graphicx}
%      \newcommand\myworkflow{dvipdfmx}  % set the flag
%    \fi
%  \fi
%\fi

\usepackage{fancyhdr}  				% Extra package for putting the watermark
% (Occupy central header)
% Users who do not need watermark can still utilize this package to produce the desired header and footer
%
% Start fancy header/footer package
\pagestyle{fancy}
\fancyhead{}  						% reset left, central, right header to empty
\fancyfoot[C]{\thepage} 			% middle footer for page number
\renewcommand{\headrulewidth}{0pt} 	% header straight line; 0pt (no line)

% If you need the watermark, please uncomment the following '\input' line
%% 浮水印 >>> 
%\input{watermark/ntust_watermark.tex}
%% <<< 浮水印

%
% global page layout
\newcommand{\mybaselinestretch}{1.5}  					% Line space: 1.5 times + 20%, (approximately double space)
\renewcommand{\baselinestretch}{\mybaselinestretch}  	% Thesis line space presets
\parskip=2ex  						% Spacing between paragraphs is equal to two heights of x
\parindent = 24Pt  					% The retraction of first paragraph is controlled by CJK, this setting cancel it

%%%%%%%%%%%%%%%%%%%%%%%%%%%%%
%  end of preamble
%%%%%%%%%%%%%%%%%%%%%%%%%%%%%		% Basic environment settings, no need to alter 基本的環境設定 無需改變  

% Personal packages, you may add or delete it as necessary 自己需要的package,可以需要增刪
\usepackage{subfig}
\usepackage{tabularx}
\usepackage{amsfonts}
\usepackage{comment}
\usepackage{slashbox}
\usepackage{rotating}
\usepackage{booktabs,threeparttable}
\renewcommand*{\thefootnote}{\fnsymbol{footnote}}
\usepackage{pdfpages}
%\usepackage{bold-extra}

% The following package is for activating hyperlinks
%	across contents, sections, figures, and tables (e.g., \ref{} and \cite{}).
%	You can comment it if you do not need this hyperlink feature.
\usepackage{hyperref}

\begin{document}



	% Following are the term definition of the thesis in English

\newcommand{\nameInnerCover}{Recommendation Letter}
\newcommand{\nameCommitteeForm}{Approval Letter}
%\newcommand{\nameCopyrightForm}{Letter of Authority}
% It's now no longer necessary to put the letter of authority,
%   both NTUST and National Library in the thesis.
%   We just need to hand on the letter separately
%   to NTUST library during the thesis submission.
%\newcommand{\nameCabstract}{Abstract in Chinese}
%\newcommand{\nameEabstract}{Abstract in English}
% Foreign students are not required to write the Chinese abstract.
%   So, we need write only one abstract in English.
\newcommand{\nameEabstract}{Abstract}
\newcommand{\nameAckn}{Acknowledgements}
\newcommand{\nameToc}{Contents}
\newcommand{\nameLot}{List of Tables }
\newcommand{\nameTof}{List of Figures}
\newcommand{\nameToa}{List of Algorithms}
\newcommand{\nameSlist}{Symbol}
\newcommand{\nameRef}{References}
\newcommand{\nameVita}{Biography}

%\renewcommand\prechaptername{Chapter} 		% It will be printed as「Chapter x」
%\renewcommand\postchaptername{}
%\renewcommand{\tablename}{Table} 			% In the table caption it will be printed as「Table x」
%\renewcommand{\figurename}{Figure} 		% In the figure caption it will be printed as「Figure x」	% Managing all Chinese vocabulary translations 在此檔案處定義文章中的中文名詞

	%----------------------------------------------------------------------------------------------------------------------------------------------------------
	%%% Following is the structure of front pages, main thesis, and back pages 以下是載入前頁、本文、後頁
	% Please do not change anything 此行請勿更動

	%----------------------------------------------------------------------------------------------------------------------------------------------------------
	% Front pages 前頁
	% Including: cover, title page, Chinese abstract, English abstract, 
	% 	acknowledgements, table of contents, list of figures, list of tables, etc.
	% You may comment out this front pages line while editing the main body of thesis,
	%	thus you can save much compilation time.
	% The structure of front pages for NTUST thesis is fixed,
	%	so do not alter the file ntust_frontpages.tex structure whatsoever,
	%	basically it consists my_names.text, my_cabstract.tex, my_eabstract.tex, my_ackn.tex, my_symbols.tex

	% 包括封面、書名頁、中文摘要、英文摘要、誌謝、目錄、表目錄、圖目錄、符號說明
	% 在撰寫各章草稿時,可以把此部份「關掉」,以節省無謂的編譯時間。
	% 實際內容由
	%    my_names.tex, my_cabstract.tex, my_eabstract.tex, my_ackn.tex, my_symbols.tex
	% 決定
	% ntust_frontpages.tex 此檔只提供整體架構的定義,不需更動
	% 在撰寫各章草稿時,可以把此部份「關掉」,以節省無謂的編譯時間。
	
	%
% This file is encoded in utf-8.
% v1.7
% Do not change the content of this file,
% 	unless the thesis layout rule is changed.
%
% 無須修改本檔內容,除非校方修改了
% 封面、書名頁、中文摘要、英文摘要、誌謝、目錄、表目錄、圖目錄、符號說明
% 等頁之格式

\renewcommand{\baselinestretch}{\mybaselinestretch}		% It makes the line spacing in effect.
\large 													% It needs a font size changing command to be effective.

% Default variables definitions
% Please notice, the following values are just default values.
% 	Please do not modify them.
%	Instead, modify the contents of my_names.tex.
%
% 注意!!此處只是預設值,不需更改此處
% 請更改 my_names.tex 內容
%
\newcommand\cTitle{論文題目}
\newcommand\eTitle{MY THESIS TITLE}
\newcommand\myCname{福蘭呢}
\newcommand\myEname{Fulan}
\newcommand\myStudentID{M1065803}
\newcommand\advisorCnameA{南宮明博士}
\newcommand\advisorEnameA{Dr.~Ming Nangong}
\newcommand\advisorCnameB{李斯坦博士}
\newcommand\advisorEnameB{Dr.~Stein Lee}
\newcommand\advisorCnameC{徐 石博士}
\newcommand\advisorEnameC{Dr.~Sean~Hsu}
\newcommand\univCname{國立台灣科技大學}
\newcommand\univEname{National Taiwan University of science and technology}
\newcommand\deptCname{光電工程研究所}
\newcommand\fulldeptEname{Graduate School of Electro-Optical Engineering}
\newcommand\deptEname{Electro-optical Engineering}
\newcommand{\DeptEname}{\expandafter\MakeUppercase\expandafter{\deptEname}}
\newcommand\collEname{College of Engineering}
\newcommand\degreeCname{碩士}
\newcommand\degreeEname{Master of Science}
\newcommand\cYear{九十四}
\newcommand\cMonth{六}
\newcommand\cDay{十}
\newcommand\eYear{2006}
\newcommand\eMonth{June}
\newcommand\ePlace{Chungli, Taoyuan, Taiwan}

% Following TeX input is to replace above default variable definitions
%
% This file is encoded in utf-8.
% v1.7
% Enter your thesis title, your name, and your personal information.
% If the title needs to use math symbols, please use \mbox{}.
%   For example: My Thesis Title \mbox{$\cal{H}_\infty$} and \mbox{Al$_x$Ga$_{1-x}$As}
% If your Chinese name contains only 2 characters, please put space in between.
% If you need to put space between name and name's title (Prof. , Dr. ), 
%   please use tilde '~' instead of normal space ' '.
% The default number of advisors used on this template is three (3),
%   feel free to remove any of them as necessary.
%
% 填入你的論文題目、姓名等資料
% 如果題目內有必須以數學模式表示的符號,請用 \mbox{} 包住數學模式,如下範例
% 如果中文名字是單名,與姓氏之間建議以全形空白填入,如下範例
% 英文名字中的稱謂,如 Prof. 以及 Dr.,其句點之後請以不斷行空白~代替一般空白,如下範例
% 如果你的指導教授沒有如預設的三位這麼多,則請把相對應的多餘教授的中文、英文名
%    的定義以空的大括號表示
%    如,\renewcommand\advisorCnameB{}
%          \renewcommand\advisorEnameB{}
%          \renewcommand\advisorCnameC{}
%          \renewcommand\advisorEnameC{}

% Thesis title (Chinese)
% 論文題目 (中文)
% It's not necessary for foreign students to include the Chinese title on the thesis
\renewcommand\cTitle{
%可驗證之安全系統的應用
}

% Thesis title (Chinese)
% 論文題目 (英文)
\renewcommand\eTitle{  
A Verifiable Secure protocol in a Secure System 
\mbox{$\cal{H}_\infty$} and \mbox{Al$_x$Ga$_{1-x}$As}
}

% My full name (Chinese)
% 我的姓名 (中文)
% It's not necessary for foreign students to include the Chinese name
%\renewcommand\myCname{王大明}
\renewcommand\myCname{Da-Ming Wang}

% My full name (English)
% 我的姓名 (英文)
\renewcommand\myEname{Da-Ming Wang}

% My student ID
%我的學號
\renewcommand\myStudentID{M1915001}

% My adviser's full name (Chinese)
% 指導教授A的姓名 (中文)
\renewcommand\advisorCnameA{陳明明 博士}
\renewcommand\advisorCnameB{}
\renewcommand\advisorCnameC{}

% My adviser's full name (English)
% 指導教授A的姓名 (英文)
\renewcommand\advisorEnameA{Dr.~Ming-Ming Cheng}
\renewcommand\advisorEnameB{}
\renewcommand\advisorEnameC{}

% Campus' name (Chinese)
% 校名 (中文)
\renewcommand\univCname{國立台灣科技大學}

% Campus' name (English)
% 校名 (英文)
\renewcommand\univEname{National Taiwan University of Science and Technology}

% Department's name (Chinese)
% 系所名 (中文)
\renewcommand\deptCname{資~訊~工~程~系}

% Department's name (English)
% 系所全名 (英文)
\renewcommand\fulldeptEname{Graduate School of Electro-Optical Engineering}

% Department's short name (English)
% 系所短名 (英文, 用於書名頁學位名領域)
\renewcommand\deptEname{Electro-Optical Engineering}

% College's name (English)
% 學院英文名 (如無,則以空的大括號表示)
\renewcommand\collEname{College of Electrical and Communication Engineering}

% Degree's name (Chinese)
% 學位名 (中文)
\renewcommand\degreeCname{碩士}
%\renewcommand\degreeCname{博士}

% Degree's name (English)
% 學位名 (英文)
\renewcommand\degreeEname{Master of Science}
%\renewcommand\degreeEname{Doctor}

% Oral exam year (Chinese, Taiwanese Year)
% 口試年份 (中文、民國)
\renewcommand\cYear{一百零七}

% Oral exam month (Chinese)
% 口試月份 (中文)
\renewcommand\cMonth{一} 

% Oral exam date (Chinese)
% 口試日份 (中文)
\renewcommand\cDay{十六} 

% Oral exam year (Arabic number, Gregorian Year)
% 口試年份 (阿拉伯數字、西元)
\renewcommand\eYear{2018} 

% Oral exam month (English)
% 口試月份 (英文)
\renewcommand\eMonth{January}

% Campus' location
% 學校所在地 (英文)
\renewcommand\ePlace{Taipei, Taiwan}

% Graduation year
%畢業級別;用於書背列印;若無此需要可忽略
\newcommand\GraduationClass{106}

%%%%%%%%%%%%%%%%%%%%%%
\newcommand\itsempty{}

%%%%%%%%%%%%%%%%%%%%%%%%%%%%%%%
%       ntust cover 封面
%%%%%%%%%%%%%%%%%%%%%%%%%%%%%%%
%
\begin{titlepage}
% no page number
% next page will be page 1
%
\begin{center}		% aligned to the center of the page

% NOTES:
% font size (relative to 12 pt):
% 	\large (14pt) < \Large (18pt) < \LARGE (20pt) < \huge (24pt) < \Huge (24 pt)

\begin{figure}[htbp]
	\begin{minipage}[b]{5cm} 
		\raggedright
		\includegraphics[width=1.1in]{frontpages/ntust_logo.pdf}
		\label{fig:ntust_logo}
	\end{minipage}% 
	\begin{minipage}[b]{0.5\textwidth} 
	\centering
	\makebox[3cm][c]{\Huge{\univCname}}\\	% Display school's Chinese name 顯示中文校名
	\vspace{0.5cm}
	\makebox[3cm][c]{\Huge{\deptCname}}\\	% Display department's Chinese name 顯示中文系所名
	\vspace{0.5cm}
	\end{minipage}%
\\ 
\rule{16cm}{3pt}
\end{figure}
%\hfill

\vspace{1cm}
\makebox[6cm][s]{\textbf{\Huge{\degreeCname 學位論文}}}\\	% Display thesis category (in Chinese) 顯示論文種類 (中文)
\vspace{1cm}

\renewcommand{\baselinestretch}{1}		% Set the line spacing to single for the titles (to compress the lines) 行距 1 倍

%\Large{\cTitle}\\	% Chinese title 中文題目
%
\vspace{1cm}
%
\Large{\eTitle}\\	% English title 英文題目
\vspace{5cm}

% \makebox is a text box with specified width;
% 	option s: stretch
% 	use \makebox to make sure 「研究生:」 and「指導教授:」 occupy the same width
%
\hspace{4.5cm} \makebox[3cm][s]{\Large{研 究 生:}}
\Large{\myCname}	% Display author's Chinese name 顯示作者中文名
\hfill \makebox[1cm][s]{}\\
%
\vspace{0.3cm}
\hspace{4.5cm} \makebox[3cm][s]{\Large{學號:}}
\Large{\myStudentID}	% Display student's ID 顯示學號
\hfill \makebox[1cm][s]{}\\
%
\vspace{1cm}
\hspace{4.5cm} \makebox[3cm][s]{\Large{指導教授:}}
\Large{\advisorCnameA}	% Display 1st adviser's name 顯示指導教授 A 中文名
\hfill \makebox[1cm][s]{}\\
%
\ifx \advisorCnameB \itsempty	% Check if the 2nd adviser exists 判斷是否有共同指導的教授 B
	% If there is no 2nd adviser,
	%	then no need to occupy the whole space,
	%	no need to print any white spaces
	% 沒有 B 教授,所以不佔版面,不印任何空白
	\relax
\else
	% If there is a 2nd adviser
	% 共同指導的教授 B
	\hspace{4.5cm} \makebox[3cm][s]{}
	\Large{\advisorCnameB}  	% Display the 2nd adviser 顯示指導教授 B 中文名
	\hfill \makebox[1cm][s]{}\\
\fi
%
\ifx \advisorCnameC \itsempty
	\relax
\else
	\hspace{4.5cm} \makebox[3cm][s]{}
	\Large{\advisorCnameC}
	\hfill \makebox[1cm][s]{}\\
\fi
%
\vfill
\makebox[10cm][s]{\Large{中華民國\cYear 年\cMonth 月\cDay 日}}%
%
\end{center}
\renewcommand{\baselinestretch}{\mybaselinestretch}   % Resume the line spacing to the desired setting 恢復原設定
% It needs a font size changing command to be effective
% Restore the font size to normal
\normalsize
\end{titlepage}


%%%%%%%%%%%%%%%%%%%
%% 標題頁
%\newpage
%\begin{titlepage}
%% aligned to the center of the page
%\begin{center}
%\Large{\eTitle}\\
%\vspace{2.8cm}
%\normalsize{By}\\
%\large{\myEname}\\
%\vspace{2.8cm}

%\normalsize{A DISSERTATION SUBMITTED IN PARTIAL FULFILLMENT OF THE}\\ 
%\normalsize{REQUIREMENTS FOR THE DEGREE OF}\\
%\vspace{1cm}
%\normalsize{\scshape DOCTOR OF PHILOSOPHY}\\
%%\normalsize{{\scshape (COMPUTER SCIENCE AND INFORMATION ENGINEERING)}}\\
%\normalsize{(\DeptEname)}\\
%\vspace{1.2cm}
%\normalsize{at the}\\
%\textsc{NATIONAL TAIWAN UNIVERSITY OF SCIENCE AND TECHNOLOGY}\\

%\vspace{6cm}
%\normalsize{\copyright Copyright by \myEname ~ 2017}
%\end{center}
%\end{titlepage}
%%%%%%%%%%%%%%%

%% 從摘要到本文之前的部份以小寫羅馬數字印頁碼
% 但是從「書名頁」(但不印頁碼) 就開始計算
\setcounter{page}{1}
\pagenumbering{roman}
%\pagenumbering{arabic}
%%%%%%%%%%%%%%%%%%%%%%%%%%%%%%%
%       指導教授推薦書 
%%%%%%%%%%%%%%%%%%%%%%%%%%%%%%%
%
% Insert the printed standard form when the thesis is ready to bind
% 在口試完成後,再將已簽名的推薦書放入以便裝訂
% Create an entry in table of contents for 推薦書
% 目前送出空白頁
\newpage{\thispagestyle{empty}\addcontentsline{toc}{chapter}{\nameInnerCover}\mbox{}\clearpage}
%\newpage

% Checking whether need to print out the watermark 判斷是否要浮水印?
\ifx\mywatermark\undefined 
  \thispagestyle{empty}  % 無頁碼、無 header (無浮水印)
\else
  \thispagestyle{EmptyWaterMarkPage} % 無頁碼、有浮水印
\fi

%%%%%%%%%%%%%%%%%%%%%%%%%%%%%%%%%%%%%%%%%%%%%%%%%%%%%%%%%%%%%%%
%% No page number
%% Create an entry in table of contents for 書名頁
%\addcontentsline{toc}{chapter}{\nameInnerCover}
%
%
%% Aligned to the center of the page
%\begin{center}
%% Font size (relative to 12 pt):
%% \large (14pt) < \Large (18pt) < \LARGE (20pt) < \huge (24pt)< \Huge (24 pt)
%% Set the line spacing to single for the titles (to compress the lines)
%\renewcommand{\baselinestretch}{1}   %行距 1 倍
%% It needs a font size changing command to be effective
%%中文題目
%\Large{\cTitle}\\ %%%%%
%\vspace{1cm}
%% 英文題目
%\Large{\eTitle}\\ %%%%%
%%\vspace{1cm}
%\vfill
%% \makebox is a text box with specified width;
%% option s: stretch
%% use \makebox to make sure
%% 「研究生:」 與「指導教授:」occupy the same width
%\makebox[3cm][s]{\large{研 究 生:}}
%\makebox[3cm][l]{\large{\myCname}} %%%%%
%\hfill
%\makebox[2cm][s]{\large{Student: }}
%\makebox[5cm][l]{\large{\myEname}}\\ %%%%%
%%
%%\vspace{1cm}
%%
%\makebox[3cm][s]{\large{指導教授:}}
%\makebox[3cm][l]{\large{\advisorCnameA}} %%%%%
%\hfill
%\makebox[2cm][s]{\large{Advisor: }}
%\makebox[5cm][l]{\large{\advisorEnameA}}\\ %%%%%
%%
%% 判斷是否有共同指導的教授 B
%\ifx \advisorCnameB  \itsempty
%\relax % 沒有 B 教授,所以不佔版面,不印任何空白
%\else
%%共同指導的教授B
%\makebox[3cm][s]{}
%\makebox[3cm][l]{\large{\advisorCnameB}} %%%%%
%\hfill
%\makebox[2cm][s]{}
%\makebox[5cm][l]{\large{\advisorEnameB}}\\ %%%%%
%\fi
%%
%% 判斷是否有共同指導的教授 C
%\ifx \advisorCnameC  \itsempty
%\relax % 沒有 C 教授,所以不佔版面,不印任何空白
%\else
%%共同指導的教授C
%\makebox[3cm][s]{}
%\makebox[3cm][l]{\large{\advisorCnameC}} %%%%%
%\hfill
%\makebox[2cm][s]{}
%\makebox[5cm][l]{\large{\advisorEnameC}}\\ %%%%%
%\fi
%%
%% Resume the line spacing to the desired setting
%\renewcommand{\baselinestretch}{\mybaselinestretch}   %恢復原設定
%\large %it needs a font size changing command to be effective
%%
%\vfill
%\makebox[4cm][s]{\large{\univCname}}\\% 校名
%\makebox[6cm][s]{\large{\deptCname}}\\% 系所名
%\makebox[3cm][s]{\large{\degreeCname 論文}}\\% 學位名
%%
%%\vspace{1cm}
%\vfill
%\large{A Thesis}\\%
%\large{Submitted to }%
%%
%\large{\fulldeptEname}\\%系所全名 (英文)
%%
%%
%\ifx \collEname  \itsempty
%\relax % 沒有學院名 (英文),所以不佔版面,不印任何空白
%\else
%% 有學院名 (英文)
%\large{\collEname}\\% 學院名 (英文)
%\fi
%%
%\large{\univEname}\\%校名 (英文)
%%
%\large{in Partial Fulfillment of the Requirements}\\
%%
%\large{for the Degree of}\\
%%
%\large{\degreeEname}\\%學位名(英文)
%
%\large{in}\\
%%
%\large{\deptEname}\\%系所短名(英文;表明學位領域)
%%
%\large{\eMonth\ \eYear}\\%月、年 (英文)
%%
%\large{\ePlace}% 學校所在地 (英文)
%\vfill
%\large{中華民國}%
%\large{\cYear}% %%%%%
%\large{年}%
%\large{\cMonth}% %%%%%
%\large{月}\\
%\end{center}
%% restore the font size to normal
%\normalsize
%\clearpage


%%%%%%%%%%%%%%%%%%%%%%%%%%%%%%%%%%%%%%%%%%%%%%%%%%%%%%%%%%%%%%%%%%%%%
%%%%%%%%%%%%%%%%%%%%%%%%%%%%%%%
%       論文口試委員審定書 (計頁碼,但不印頁碼) 
%%%%%%%%%%%%%%%%%%%%%%%%%%%%%%%
%
% Insert the printed standard form when the thesis is ready to bind
% 在口試完成後,再將已簽名的審定書放入以便裝訂
% Create an entry in table of contents for 審定書
% 目前送出空白頁
\newpage{\thispagestyle{empty}\addcontentsline{toc}{chapter}{\nameCommitteeForm}\mbox{}\clearpage}


%%%%%%%%%%%%%%%%%%%%%%%%%%%%%%%
%       中文摘要 
%%%%%%%%%%%%%%%%%%%%%%%%%%%%%%%
%
%\newpage
%\thispagestyle{plain}  % 無 header,但在浮水印模式下會有浮水印
% Create an entry in table of contents for 中文摘要
%\addcontentsline{toc}{chapter}{\nameCabstract}

% Aligned to the center of the page
%\begin{center}
% Font size (relative to 12 pt):
% \large (14pt) < \Large (18pt) < \LARGE (20pt) < \huge (24pt)< \Huge (24 pt)
% Set the line spacing to single for the names (to compress the lines)
%\renewcommand{\baselinestretch}{1}   %行距 1 倍
% It needs a font size changing command to be effective
%\begin{comment}
%\large{\cTitle}\\  %中文題目
%\vspace{0.83cm}
% \makebox is a text box with specified width;
% 	Option s: stretch
% 	Use \makebox to make sure
% 	Each text field occupies the same width
%\makebox[1.5cm][s]{\large{學生:}}
%\makebox[3cm][l]{\large{\myCname}} %學生中文姓名
%\hfill
%
%\makebox[3cm][s]{\large{指導教授:}}
%\makebox[3cm][l]{\large{\advisorCnameA}} \\ %教授A中文姓名
%
% 判斷是否有共同指導的教授 B
%\ifx \advisorCnameB  \itsempty
%\relax % 沒有 B 教授,所以不佔版面,不印任何空白
%\else
%共同指導的教授B
%\makebox[1.5cm][s]{}
%\makebox[3cm][l]{} %%%%%
%\hfill
%\makebox[3cm][s]{}
%\makebox[3cm][l]{\large{\advisorCnameB}}\\ %教授B中文姓名
%\fi
%
% 判斷是否有共同指導的教授 C
%\ifx \advisorCnameC  \itsempty
%\relax % 沒有 C 教授,所以不佔版面,不印任何空白
%\else
%共同指導的教授C
%\makebox[1.5cm][s]{}
%\makebox[3cm][l]{} %%%%%
%\hfill
%\makebox[3cm][s]{}
%\makebox[3cm][l]{\large{\advisorCnameC}}\\ %教授C中文姓名
%\fi
%
%\vspace{0.42cm}
%
%\large{\univCname}\large{\deptCname}\\ %校名系所名
%\vspace{0.83cm}
%\vfill
%\end{comment}
%\makebox[2.5cm][s]{\Large{摘要}}\\
%\end{center}
% Resume the line spacing to the desired setting
%\renewcommand{\baselinestretch}{\mybaselinestretch}   %恢復原設定
% It needs a font size changing command to be effective
% Restore the font size to normal
%\normalsize
%%%%%%%%%%%%%
%\input{frontpages/my_cabstract.tex}

%%%%%%%%%%%%%%%%%%%%%%%%%%%%%%%
%       英文摘要 
%%%%%%%%%%%%%%%%%%%%%%%%%%%%%%%
%
\newpage
%\thispagestyle{plain}  % 無 header,但在浮水印模式下會有浮水印
\chapter*{\protect\makebox[3cm][s]{Abstract}}
% Create an entry in table of contents for 英文摘要
\addcontentsline{toc}{chapter}{\nameEabstract}

% Aligned to the center of the page
\begin{comment}
\begin{center}
% Font size:
% \large (14pt) < \Large (18pt) < \LARGE (20pt) < \huge (24pt)< \Huge (24 pt)
% Set the line spacing to single for the names (to compress the lines)
\renewcommand{\baselinestretch}{1}   %行距 1 倍

%\large % it needs a font size changing command to be effective
\large{\eTitle}\\  %英文題目
\vspace{0.83cm}
% \makebox is a text box with specified width;
% 	Option s: stretch
% 	Use \makebox to make sure
% 	Each text field occupies the same width
\makebox[2cm][s]{\large{Student: }}
\makebox[5cm][l]{\large{\myEname}} %學生英文姓名
\hfill
%
\makebox[2cm][s]{\large{Advisor: }}
\makebox[5cm][l]{\large{\advisorEnameA}} \\ %教授A英文姓名
%
% 判斷是否有共同指導的教授 B
\ifx \advisorCnameB  \itsempty
\relax % 沒有 B 教授,所以不佔版面,不印任何空白
\else
%共同指導的教授B
\makebox[2cm][s]{}
\makebox[5cm][l]{} %%%%%
\hfill
\makebox[2cm][s]{}
\makebox[5cm][l]{\large{\advisorEnameB}}\\ %教授B英文姓名
\fi
%
% 判斷是否有共同指導的教授 C
\ifx \advisorCnameC  \itsempty
\relax % 沒有 C 教授,所以不佔版面,不印任何空白
\else
%共同指導的教授C
\makebox[2cm][s]{}
\makebox[5cm][l]{} %%%%%
\hfill
\makebox[2cm][s]{}
\makebox[5cm][l]{\large{\advisorEnameC}}\\ %教授C英文姓名
\fi
%
\vspace{0.42cm}
\large{Submitted to }\large{\fulldeptEname}\\  %英文系所全名
%
\ifx \collEname  \itsempty
\relax % 如果沒有學院名 (英文),則不佔版面,不印任何空白
\else
% 有學院名 (英文)
\large{\collEname}\\% 學院名 (英文)
\fi
%
\large{\univEname}\\  %英文校名
\vspace{0.83cm}
%\vfill
%

%\Large{ABSTRACT}\\
%\vspace{0.5cm}
\end{center}
\end{comment}
% Resume the line spacing the desired setting
%\renewcommand{\baselinestretch}{\mybaselinestretch}   %恢復原設定
%\large %it needs a font size changing command to be effective
% restore the font size to normal
%\normalsize
%%%%%%%%%%%%%
\input{frontpages/my_eabstract.tex}

%%%%%%%%%%%%%%%%%%%%%%%%%%%%%%%
%       誌謝 
%%%%%%%%%%%%%%%%%%%%%%%%%%%%%%%
%
% Acknowledgment
\newpage
\chapter*{\protect\makebox[5cm][s]{\nameAckn}} %\makebox{} is fragile; need protect
\addcontentsline{toc}{chapter}{\nameAckn}
\textit{alHamdulillāh}. Years of accomplishment codified on this dissertation was not easy.
It is my honor to thank all people who have all along supported me graciously.

First and foremost, I would like to express my deepest gratitude to Doctor Fulan Teng,
whose encouragement, thoughtful guidance, brilliant ideas and wholehearted supports
enabled me to develop an understanding of this research subject.
His every advice has been very valuable to my own professional growth and will not be forgotten.
I would also like to extend my gratitude to 
all committee members in my doctoral defense for their valuable comments.

It is a pleasure to thank the members of Bla Bla Bla Laboratory.

Finally, I am heartily thankful to my family
for their unconditional love, continuous supports and cheers.

%%%%%%%%%%%%%%%%%%%%%%%%%%%%%%%
%       目錄 
%%%%%%%%%%%%%%%%%%%%%%%%%%%%%%%
%
% Table of contents
\newpage
\renewcommand{\contentsname}{\protect\makebox[3cm][s]{\nameToc}}
%\makebox{} is fragile; need protect
\addcontentsline{toc}{chapter}{\nameToc}
\tableofcontents



%%%%%%%%%%%%%%%%%%%%%%%%%%%%%%%
%       圖目錄 
%%%%%%%%%%%%%%%%%%%%%%%%%%%%%%%
%
% List of Figures
\newpage
\renewcommand{\listfigurename}{\protect\makebox[5cm][s]{\nameTof}}
%\makebox{} is fragile; need protect
\addcontentsline{toc}{chapter}{\nameTof}
\listoffigures

%%%%%%%%%%%%%%%%%%%%%%%%%%%%%%%
%       表目錄 
%%%%%%%%%%%%%%%%%%%%%%%%%%%%%%%
%
% List of Tables
\newpage
\renewcommand{\listtablename}{\protect\makebox[5cm][s]{\nameLot}}
%\makebox{} is fragile; need protect
\addcontentsline{toc}{chapter}{\nameLot}
\listoftables

%%%%%%%%%%%%%%%%%%%%%%%%%%%%%%%
%       演算法目錄 
%%%%%%%%%%%%%%%%%%%%%%%%%%%%%%%
%
% List of Figures
\newpage
\renewcommand{\listalgorithmname}{\protect\makebox[5cm][s]{\nameToa}}
%\makebox{} is fragile; need protect
\addcontentsline{toc}{chapter}{\nameToa}
\listofalgorithms


%%%%%%%%%%%%%%%%%%%%%%%%%%%%%%%
%       符號說明 
%%%%%%%%%%%%%%%%%%%%%%%%%%%%%%%
%
% Symbol list
% 	Define new environment, based on standard description environment
% 	adapted from p.60~64, <<The LaTeX Companion>>, 1994, ISBN 0-201-54199-8
%\newcommand{\SymEntryLabel}[1]%
% {\makebox[3cm][l]{#1}}
%
%\newenvironment{SymEntry}
%   {\begin{list}{}%
%       {\renewcommand{\makelabel}{\SymEntryLabel}%
%        \setlength{\labelwidth}{3cm}%
%        \setlength{\leftmargin}{\labelwidth}%
%        }%
%   }%
%   {\end{list}}
%%
%\newpage
%\chapter*{\protect\makebox[5cm][s]{\nameSlist}} %\makebox{} is fragile; need protect
%\addcontentsline{toc}{chapter}{\nameSlist}
%\input{frontpages/my_symbols.tex}


\newpage
%% 論文本體頁碼回復為阿拉伯數字計頁,並從頭起算
\pagenumbering{arabic}
%%%%%%%%%%%%%%%%%%%%%%%%%%%%%%%% 

	%----------------------------------------------------------------------------------------------------------------------------------------------------------
	% Main body 論文主體
	% You are suggested to separate chapters into different files. 
	% 	For example: introduction.tex, related_work.tex, conclusion.tex
	%	Then put the files under sub folder "section/"

	% 建議以「章」為檔案分割的依據。
	% 以下為建議的命名分類
	%   introduction.tex, related_work.tex, conclusion.tex
	% 做為這幾個「章」的檔案名稱,並將檔案存放於資料夾 sections/ 下
	% 實際命名方式可以隨你意
	% 在撰寫各章草稿時,可以把其他章節關掉 (行首加百分號)

	\chapter{Introduction}
\label{cha:1_introduction} 

Security in wireless sensor networks (WSNs) has become a popular research field in recent years, and node identification is considered as one of the most important issues in this field~\cite{MaddenFHH02}. In WSNs, the mechanism to create and manage node identities is usually naive and is not well protected. Thus many attack techniques, such as Sybil attacks and replication attacks, are used to exploit this vulnerability. 

	Since the node identities are easy to create and change, a reliable node identification mechanism is needed in sensor networks. Currently several authentication and certification methods have been proposed to ensure the node identification. However, these approaches use cryptographic techniques, and thus inevitably increase computing overhead of sensor nodes. This chapter introduces a simple but effective method to identify a node only by measuring its clock skew.

	Recently, Chen et al. revealed the possibility to fingerprint every computer in general networks by their clock skews. Murdoch's research also used clock skew as a main method to detect the identities behind the Tor network. However, there are few studies evaluating the characteristics of clock skew in WSNs~\cite{GarofalakisHM07}. In this research, we use the Flooding Time Synchronization Protocol (FTSP) to measure the time information of each mote, and successfully observe that every sensor mote does have constant and unique clock skew~\cite{KotiVDSD07, SubrPPKG06, ShengLMJ07, Wagner04}. An algorithm to group and identify clock skews of large amount of motes is proposed, and its applications like Sybil attack detection are also discussed in Table~\ref{tab:1_system}.
	
	Generally, there are two steps to measure the clock skew between two devices.
	The first step is to collect the timestamp from the sender via a certain protocol.
	After collecting enough timestamp, the receiver will apply a clock skew estimation algorithm (such as linear regression, linear programming or piecewise minimum), to calculate the clock skew in the second step.
	Due to different network environments, we need to use different protocols and estimation algorithms to calculate clock skews.
	Since we will apply clock skew device identification to different networks, such as wireless sensor networks and cloud environment, more detailed procedures will be discussed in each chapter.


\begin{table}[t!]
  \begin{center}
    \caption{The relation of aggregation overhead between different techniques}
    \label{tab:1_system}
    \begin{tabular}{|c|c c c|}
      \hline
       & Space usage & Communication & Query \\
       & of root aggregator & overhead & requirement \\
      \hline
      Traditional warehouse & $n$ & $O(n)$ & $O(n)$ \\
      \hline
      AM-FM sketch technique & $\log a$ & $O(\log n)$ &  $O(a\log n)$ \\
      \hline
      ``prototypical PHI query'' & $\log a$ & $O(\log n)$ & $O(\log n)$ \\
      \hline
      \end{tabular}
  \end{center}
\end{table}
	\chapter{Method}
\label{cha:2_method}
\section{Preliminaries}
\label{sec:2_preliminaries}

With the rapid growth in integrated circuit, digital signal processing, and other emerging technologies, people nowadays can easily purchase electronic devices, such as personal computers, laptops, cellular phones, and tablets.  
By utilizing these devices, people can communicate with each other through wireless communication and increase work performance.      
However, any malicious user may misuse these devices and launch serious attack to make illegal profit, such as identity stealing or password cracking on a bank account.
Therefore, it is essential to develop robust methods to solve the identity problems.
With the rapid growth in integrated circuit, digital signal processing, and other emerging technologies, people nowadays can easily purchase electronic devices, such as personal computers, laptops, cellular phones, and tablets.  
By utilizing these devices, people can communicate with each other through wireless communication and increase work performance.      
However, any malicious user may misuse these devices and launch serious attack to make illegal profit, such as identity stealing or password cracking on a bank account.
Therefore, it is essential to develop robust methods to solve the identity problems.
With the rapid growth in integrated circuit, digital signal processing, and other emerging technologies, people nowadays can easily purchase electronic devices, such as personal computers, laptops, cellular phones, and tablets.  
By utilizing these devices, people can communicate with each other through wireless communication and increase work performance.      
However, any malicious user may misuse these devices and launch serious attack to make illegal profit, such as identity stealing or password cracking on a bank account.
Therefore, it is essential to develop robust methods to solve the identity problems, as shown in
~\ref{fig:2_PHI}.

\begin{figure}[t!]
  \begin{center}
    \includegraphics[width=1.0\textwidth]{figures/dyna_rm.pdf}
    \caption{The diagram of ``prototypical PHI query''} 
    \label{fig:2_PHI}
  \end{center}
\end{figure}
	
	\chapter{Conclusions}
\label{cha:3_conclusions}

With the rapid growth in integrated circuit, digital signal processing, and other emerging technologies, people nowadays can easily purchase electronic devices, such as personal computers, laptops, cellular phones, and tablets.  
By utilizing these devices, people can communicate with each other through wireless communication and increase work performance.      
However, any malicious user may misuse these devices and launch serious attack to make illegal profit, such as identity stealing or password cracking on a bank account.
Therefore, it is essential to develop robust methods to solve the identity problems.

\section{Future Work} 
\label{sec:3_future_work}
With the rapid growth in integrated circuit, digital signal processing, and other emerging technologies, people nowadays can easily purchase electronic devices, such as personal computers, laptops, cellular phones, and tablets.  
By utilizing these devices, people can communicate with each other through wireless communication and increase work performance.      
However, any malicious user may misuse these devices and launch serious attack to make illegal profit, such as identity stealing or password cracking on a bank account.
Therefore, it is essential to develop robust methods to solve the identity problems.


	%----------------------------------------------------------------------------------------------------------------------------------------------------------
	% Back pages 後頁
	% Including: references, appendixes, and curriculum vitae
	% The structure of back pages for NTUST thesis is fixed,
	%	so do not alter the file ntust_backpages.tex structure whatsoever.
	% You can also comment out this back pages line while editing the main body of thesis.

	% 包括參考文獻、附錄、自傳
	% 實際內容由
	%    my_bib.bib, my_appendix.tex, my_vita.tex
	% 決定
	% ntust_backpages.tex 此檔只提供整體架構的定義,不需更動
	% 在撰寫各章草稿時,可以把此部份「關掉」,以節省無謂的編譯時間。

	%
% This file is encoded in utf-8
% v1.7
%

%%% References Document/ Bibliography 
\newpage
\addcontentsline{toc}{chapter}{\nameRef}
\renewcommand{\bibname}{\protect\makebox[5cm][s]{\nameRef}}		% '\makebox{}' is fragile, '\protect' will help
\bibliographystyle{ieeetr}										% Using IEEE Trans journal format
{\footnotesize
\bibliography{my_bib}  											% 'my_bib' is a BibTeX file located in the parent directory, it comprises all the references for the thesis
}

%%% Appendix (uncomment the following line to include the appendixes)
%\input{backpages/my_appendix.tex}

%%% Autobiography/ Vitae
%\newpage
%\chapter*{\protect\makebox[5cm][s]{\nameVitae}} 				% '\makebox{}' is fragile, '\protect' will help
%\addcontentsline{toc}{chapter}{\nameVitae}
%Fulan Chang is currently a Ph.D candidate 
in Department of Computer Science and Information Engineering 
from National Taiwan University of Science and Technology, Taipei, Taiwan. 
He received M.S degree in Department of Computer Science  and Information Engineering 
from National Taiwan University of Science and Technology in 2015. 
His current research interests include the Human-Computer Interaction, 
Tele-operation and Robotics. 



%%%%%%%%%%%%%%%%%%%%%%%%%%%%%%%%%%%%%%%%%%%%%%%%%%%%%%%%%%%%%%
%   Letter of Authorization (LoA/ 授權書)
%     ... the page numbers will be counted but not be printed
%%%%%%%%%%%%%%%%%%%%%%%%%%%%%%%%%%%%%%%%%%%%%%%%%%%%%%%%%%%%%%
%
% Insert the printed standard form when the thesis is ready to bind
% After the oral defense, put the LoA (signed by advisor) into the thesis, then bind it
% Create an entry in table of contents for this LoA
% Currently only generate an empty page

% Per January 2018, there is no need to include the copyright page
%   into the thesis book.
%   The copyright (authorization) letters are submitted separately to NTUST library.

%\newpage{\thispagestyle{empty}\addcontentsline{toc}{chapter}{\nameCopyrightForm}\mbox{}\clearpage}

%\includepdf{backpages/a.pdf}{\addcontentsline{toc}{chapter}{\nameCopyrightForm}}


%%%%% useless %%%%%
%\begin{figure}[!t]
%\centering
%\includegraphics[width=\paperwidth]{backpages/LetterofAuthority.pdf}
%\end{figure}
%\includepdf[addtotoc={1,chapter,-1,\nameCopyrightForm,letter}]{backpages/LetterofAuthority.pdf}
%\includepdf{backpages/LetterofAuthority.pdf}
%\addcontentsline{toc}{chapter}{\nameCopyrightForm}\mbox{}

\end{document} 
 
